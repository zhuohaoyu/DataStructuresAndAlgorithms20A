\PassOptionsToPackage{unicode=true}{hyperref} % options for packages loaded elsewhere
\PassOptionsToPackage{hyphens}{url}
%
\documentclass{article}
\usepackage[a4paper, total={6in, 8in}]{geometry}
\usepackage{ctex}
\usepackage{lmodern}
\usepackage{amssymb,amsmath}
\usepackage{ifxetex,ifluatex}
\usepackage{fixltx2e} % provides \textsubscript
\usepackage{unicode-math}
\defaultfontfeatures{Ligatures=TeX,Scale=MatchLowercase}
\usepackage[]{microtype}
\UseMicrotypeSet[protrusion]{basicmath} % disable protrusion for tt fonts
\IfFileExists{parskip.sty}{%
\usepackage{parskip}
}{% else
\setlength{\parindent}{0pt}
\setlength{\parskip}{6pt plus 2pt minus 1pt}
}
\usepackage{hyperref}
\hypersetup{
            pdfborder={0 0 0},
            breaklinks=true}
\urlstyle{same}  % don't use monospace font for urls
\usepackage{color}
\usepackage{fancyvrb}
\newcommand{\VerbBar}{|}
\newcommand{\VERB}{\Verb[commandchars=\\\{\}]}
\DefineVerbatimEnvironment{Highlighting}{Verbatim}{commandchars=\\\{\}}
% Add ',fontsize=\small' for more characters per line
\newenvironment{Shaded}{}{}
\newcommand{\AlertTok}[1]{\textcolor[rgb]{1.00,0.00,0.00}{\textbf{#1}}}
\newcommand{\AnnotationTok}[1]{\textcolor[rgb]{0.38,0.63,0.69}{\textbf{\textit{#1}}}}
\newcommand{\AttributeTok}[1]{\textcolor[rgb]{0.49,0.56,0.16}{#1}}
\newcommand{\BaseNTok}[1]{\textcolor[rgb]{0.25,0.63,0.44}{#1}}
\newcommand{\BuiltInTok}[1]{#1}
\newcommand{\CharTok}[1]{\textcolor[rgb]{0.25,0.44,0.63}{#1}}
\newcommand{\CommentTok}[1]{\textcolor[rgb]{0.38,0.63,0.69}{\textit{#1}}}
\newcommand{\CommentVarTok}[1]{\textcolor[rgb]{0.38,0.63,0.69}{\textbf{\textit{#1}}}}
\newcommand{\ConstantTok}[1]{\textcolor[rgb]{0.53,0.00,0.00}{#1}}
\newcommand{\ControlFlowTok}[1]{\textcolor[rgb]{0.00,0.44,0.13}{\textbf{#1}}}
\newcommand{\DataTypeTok}[1]{\textcolor[rgb]{0.56,0.13,0.00}{#1}}
\newcommand{\DecValTok}[1]{\textcolor[rgb]{0.25,0.63,0.44}{#1}}
\newcommand{\DocumentationTok}[1]{\textcolor[rgb]{0.73,0.13,0.13}{\textit{#1}}}
\newcommand{\ErrorTok}[1]{\textcolor[rgb]{1.00,0.00,0.00}{\textbf{#1}}}
\newcommand{\ExtensionTok}[1]{#1}
\newcommand{\FloatTok}[1]{\textcolor[rgb]{0.25,0.63,0.44}{#1}}
\newcommand{\FunctionTok}[1]{\textcolor[rgb]{0.02,0.16,0.49}{#1}}
\newcommand{\ImportTok}[1]{#1}
\newcommand{\InformationTok}[1]{\textcolor[rgb]{0.38,0.63,0.69}{\textbf{\textit{#1}}}}
\newcommand{\KeywordTok}[1]{\textcolor[rgb]{0.00,0.44,0.13}{\textbf{#1}}}
\newcommand{\NormalTok}[1]{#1}
\newcommand{\OperatorTok}[1]{\textcolor[rgb]{0.40,0.40,0.40}{#1}}
\newcommand{\OtherTok}[1]{\textcolor[rgb]{0.00,0.44,0.13}{#1}}
\newcommand{\PreprocessorTok}[1]{\textcolor[rgb]{0.74,0.48,0.00}{#1}}
\newcommand{\RegionMarkerTok}[1]{#1}
\newcommand{\SpecialCharTok}[1]{\textcolor[rgb]{0.25,0.44,0.63}{#1}}
\newcommand{\SpecialStringTok}[1]{\textcolor[rgb]{0.73,0.40,0.53}{#1}}
\newcommand{\StringTok}[1]{\textcolor[rgb]{0.25,0.44,0.63}{#1}}
\newcommand{\VariableTok}[1]{\textcolor[rgb]{0.10,0.09,0.49}{#1}}
\newcommand{\VerbatimStringTok}[1]{\textcolor[rgb]{0.25,0.44,0.63}{#1}}
\newcommand{\WarningTok}[1]{\textcolor[rgb]{0.38,0.63,0.69}{\textbf{\textit{#1}}}}
\setlength{\emergencystretch}{3em}  % prevent overfull lines
\providecommand{\tightlist}{%
  \setlength{\itemsep}{0pt}\setlength{\parskip}{0pt}}
\setcounter{secnumdepth}{0}
% Redefines (sub)paragraphs to behave more like sections
\ifx\paragraph\undefined\else
\let\oldparagraph\paragraph
\renewcommand{\paragraph}[1]{\oldparagraph{#1}\mbox{}}
\fi
\ifx\subparagraph\undefined\else
\let\oldsubparagraph\subparagraph
\renewcommand{\subparagraph}[1]{\oldsubparagraph{#1}\mbox{}}
\fi

% set default figure placement to htbp
\makeatletter
\def\fps@figure{htbp}
\makeatother

\title{数据结构与算法I 思考题8}
\author{2019201409 于倬浩}

\begin{document}
\maketitle

\hypertarget{header-n4}{%
\subsection{16.2-6}\label{header-n4}}

将分数背包算法从\(\Theta(nlgn)\)优化到\(\Theta(n)\)的时间复杂度。

首先,改进的算法一定不能改变贪心算法的贪心策略,因此采用和\(\Theta(nlgn)\)算法相同的贪心策略,显然不需要赘述原算法贪心策略的正确性。

考虑原算法的瓶颈,在于将所有物品按照平均值(价值/重量)从大到小排序,排序的过程需要\(\Theta(nlgn)\)的复杂度,而贪心策略本身其实就是选择平均值最大的几个物品,且最后一个物品可能选到分数份。

因此,考虑使用之前学过的最坏时间复杂度\(\Theta(n)\)的线性选择中位数算法\texttt{select()}进行优化,目的依旧是找出平均值最大的若干个物品来填满背包。

首先,对于子问题\texttt{FractionalKnapsack(L,\ R,\ weight)},表示只选择下标(可能被重标号){[}L,R{]}范围内的物品,且背包容量为weight的最优解。首先线性选出当前区间的平均值的中位数,统计平均值大于等于中位数的物品的重量之和。如果重量和小于当前的上限,说明最终的答案取到的最小平均值一定更小,因此解为子问题\texttt{FractionalKnapsack(L,\ p\ -\ 1,\ weight\ -\ curweight)}的解与这些平均值大于中位数的物品的并集;如果重量和大于等于当前的上限,说明最终答案取到的最小平均值更大,也就是这些物品的一个子集,因此为子问题\texttt{FractionalKnapsack(p\ +\ 1,\ R,\ weight)}的解。边界条件是当区间长度为1时,需要返回\texttt{item{[}L{]}.avg\ *\ min(weight,\ item{[}L{]}.weight)},表示最后一个物品可以选择分数份。

\begin{Shaded}
\begin{Highlighting}[]
\DataTypeTok{double}\NormalTok{ FractionalKnapsack(}\DataTypeTok{int}\NormalTok{ L, }\DataTypeTok{int}\NormalTok{ R, }\DataTypeTok{double}\NormalTok{ weight) \{}
    \ControlFlowTok{if}\NormalTok{(weight <= }\DecValTok{0}\NormalTok{) }\ControlFlowTok{return} \DecValTok{0}\NormalTok{; }\CommentTok{//边界:背包已满}
    \ControlFlowTok{if}\NormalTok{(L == R) }\ControlFlowTok{return}\NormalTok{ item[L].avg * min(weight, item[L].weight); }
    \CommentTok{//边界:需要切分的最后一个物品}
    \DataTypeTok{int}\NormalTok{ p = select(L, R); }
    \CommentTok{//线性选择算法,找到区间[L,R]的中位数对应下标,且保证[L,p]都<=中位数,[p+1,R]>=中位数}
    \DataTypeTok{double}\NormalTok{ curweight = }\DecValTok{0}\NormalTok{, curvalue = }\DecValTok{0}\NormalTok{;}
    \ControlFlowTok{for}\NormalTok{(}\DataTypeTok{int}\NormalTok{ i = p; i <= R; ++i) \{ }\CommentTok{//计算平均值大于中位数的物品的重量、价值之和}
\NormalTok{        curweight += item[i].weight;}
\NormalTok{        curvalue += item[i].value;}
\NormalTok{    \}}
    \ControlFlowTok{if}\NormalTok{(curweight <= weight) }\CommentTok{//含义如上}
        \ControlFlowTok{return}\NormalTok{ curvalue + FractionalKnapsack(L, p - }\DecValTok{1}\NormalTok{, weight - curweight);}
    \ControlFlowTok{else} 
        \ControlFlowTok{return}\NormalTok{ FractionalKnapsack(p + }\DecValTok{1}\NormalTok{, R, weight);}
\NormalTok{\}}
\end{Highlighting}
\end{Shaded}

\end{document}
