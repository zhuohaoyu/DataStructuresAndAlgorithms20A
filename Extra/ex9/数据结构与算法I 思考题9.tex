\PassOptionsToPackage{unicode=true}{hyperref} % options for packages loaded elsewhere
\PassOptionsToPackage{hyphens}{url}
%
\documentclass[]{article}
\usepackage[a4paper,top=3cm,bottom=3cm,left=3cm,right=3cm,marginparwidth=1.75cm]{geometry}
\usepackage[UTF8]{ctex}
\usepackage{lmodern}
\usepackage{amssymb,amsmath}
\usepackage{ifxetex,ifluatex}
\usepackage{fixltx2e} % provides \textsubscript
\ifnum 0\ifxetex 1\fi\ifluatex 1\fi=0 % if pdftex
  \usepackage[T1]{fontenc}
  \usepackage[utf8]{inputenc}
  \usepackage{textcomp} % provides euro and other symbols
\else % if luatex or xelatex
  \usepackage{unicode-math}
  \defaultfontfeatures{Ligatures=TeX,Scale=MatchLowercase}
\fi
% use upquote if available, for straight quotes in verbatim environments
\IfFileExists{upquote.sty}{\usepackage{upquote}}{}
% use microtype if available
\IfFileExists{microtype.sty}{%
\usepackage[]{microtype}
\UseMicrotypeSet[protrusion]{basicmath} % disable protrusion for tt fonts
}{}
\IfFileExists{parskip.sty}{%
\usepackage{parskip}
}{% else
\setlength{\parindent}{0pt}
\setlength{\parskip}{6pt plus 2pt minus 1pt}
}
\usepackage{hyperref}
\hypersetup{
            pdfborder={0 0 0},
            breaklinks=true}
\urlstyle{same}  % don't use monospace font for urls
\setlength{\emergencystretch}{3em}  % prevent overfull lines
\providecommand{\tightlist}{%
  \setlength{\itemsep}{0pt}\setlength{\parskip}{0pt}}
\setcounter{secnumdepth}{0}
% Redefines (sub)paragraphs to behave more like sections
\ifx\paragraph\undefined\else
\let\oldparagraph\paragraph
\renewcommand{\paragraph}[1]{\oldparagraph{#1}\mbox{}}
\fi
\ifx\subparagraph\undefined\else
\let\oldsubparagraph\subparagraph
\renewcommand{\subparagraph}[1]{\oldsubparagraph{#1}\mbox{}}
\fi

% set default figure placement to htbp
\makeatletter
\def\fps@figure{htbp}
\makeatother

\title{数据结构与算法I 思考题9}
\author{2019201409 于倬浩}
\begin{document}
    
\maketitle
\hypertarget{header-n2}{%
\subsection{问题}\label{header-n2}}

假设我们将斐波拉契堆的\texttt{decrease-key}操作改为:一个节点在失去\(k\)个孩子后才会被切开,其中k为大于2的常数。证明这样的改动使得\texttt{decrease-key}的摊还代价降低常数倍,而\texttt{extract-min}的摊还代价增加常数倍。

\hypertarget{header-n9}{%
\subsection{证明}\label{header-n9}}

修改势函数的定义。设\(F_i(D_j)\)表示斐波那契堆\(D_j\)中,被标记\(i\)次的节点个数。那么,势函数的计算方式修改如下:

\[\phi(D_i) = \sum_{j=1}^{k} (k - j + 1) F_j(D_i) + t(D_i)\]

那么,每次标记某个节点,会导致势函数增加1,或是减少k。一次\texttt{decrease-key}操作,假设调用了\(c\)次\texttt{cut},则必定减少了\(c-1\)个标记次数为k的节点,因此势函数必定减少\(-k \times(c-1)\),另外,最多标记了\(c\)次节点,因此又带来\(c\)的增量。另外,由于又增加了\(c - 1\)个根,因此势函数改变量最大为:

\[-k \times (c - 1) + c -1 = k - 1 - (k-1)c\]

均摊代价最多为\(O(c)+k-1-(k-1)c=O(c)+k-1=O(c)\),对比原算法的均摊代价为\(O(c)+4-c = O(c)\)。由于第一个\(O(c)\)中的常数因子没有改变,而当\(k>2\)时,\(k-1 < 1\),因此实际上减去的部分中常数因子更大,所以\texttt{decrease-key}操作会有常数上的改进。另外,尽管实际上增加了一个\(k-1\)的常数因子,由于减去的部分\((k-1)c\)还有一个系数\(c\),所以增加的常数因子比减少的要小,即实际上会有常数上的改进。

另一方面,由于\texttt{extract-min}操作中,最多有一个节点的标记数量增加,由于新的势函数保证了增量和原有算法势函数保持一致,因此实际上在这部分的势能分析和原算法类似,不再赘述,均摊代价为\(O(D(n))\),仅需考虑\(O(D(n))\)的变化即可(\(D(n)\)为节点度数)。

重新考虑我们对于节点度数的下界分析:假设当前有一个度数为\(k\)的节点x,现在我们正在插入第\(i\)个节点\(y_i\),此时保证\(x.degree \ge i - 1, \; y_i.degree \ge i - 1 - (k - 1) = i - k\),和原算法的\(y_i.degree \ge i-2\)不同。对于其余部分,分析思路和CLRS上\(Lemma \;19.1\)到\(Lemma \;19.3\)保持一致,唯一不同的是,由于\(y_i.degree >=0\)的条件变为了\(i >= k\),因此实际上递归式不再是严格的斐波那契数列,而是需要修改第\(0\)到第\(k-1\)项为常数,之后从第\(k\)项开始才能继续沿用原有递推式。使用《离散数学》的知识分析,得知这样虽然可以保证节点度数\(k \leq \lfloor log_{\phi'} n \rfloor\),但是实际上这里的\(\phi'\)需要比原来的\(\phi\)小,底数变小,基数不变,因此带来常数上的增加,即节点度数的上界增加一个常数(特征方程不变,通解不变,但由于初始条件改变,因此得到的解发生变化)。具体分析过程与本门课程无关,在此略去。

因此结论是,对于\texttt{decrease-key}操作,修改后的算法会减小常数代价,而对于\texttt{extract-min}操作,修改后的算法会导致摊还代价增加常数倍。

\end{document}
