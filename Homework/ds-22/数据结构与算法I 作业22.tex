\PassOptionsToPackage{unicode=true}{hyperref} % options for packages loaded elsewhere
\PassOptionsToPackage{hyphens}{url}
%
\documentclass[]{article}
\usepackage[a4paper,top=3cm,bottom=3cm,left=3cm,right=3cm,marginparwidth=1.75cm]{geometry}
\usepackage[UTF8]{ctex}
\usepackage{lmodern}
\usepackage{amssymb,amsmath}
\usepackage{ifxetex,ifluatex}
\usepackage{fixltx2e} % provides \textsubscript
\ifnum 0\ifxetex 1\fi\ifluatex 1\fi=0 % if pdftex
  \usepackage[T1]{fontenc}
  \usepackage[utf8]{inputenc}
  \usepackage{textcomp} % provides euro and other symbols
\else % if luatex or xelatex
  \usepackage{unicode-math}
  \defaultfontfeatures{Ligatures=TeX,Scale=MatchLowercase}
\fi
% use upquote if available, for straight quotes in verbatim environments
\IfFileExists{upquote.sty}{\usepackage{upquote}}{}
% use microtype if available
\IfFileExists{microtype.sty}{%
\usepackage[]{microtype}
\UseMicrotypeSet[protrusion]{basicmath} % disable protrusion for tt fonts
}{}
\IfFileExists{parskip.sty}{%
\usepackage{parskip}
}{% else
\setlength{\parindent}{0pt}
\setlength{\parskip}{6pt plus 2pt minus 1pt}
}
\usepackage{hyperref}
\hypersetup{
            pdfborder={0 0 0},
            breaklinks=true}
\urlstyle{same}  % don't use monospace font for urls
\setlength{\emergencystretch}{3em}  % prevent overfull lines
\providecommand{\tightlist}{%
  \setlength{\itemsep}{0pt}\setlength{\parskip}{0pt}}
\setcounter{secnumdepth}{0}
% Redefines (sub)paragraphs to behave more like sections
\ifx\paragraph\undefined\else
\let\oldparagraph\paragraph
\renewcommand{\paragraph}[1]{\oldparagraph{#1}\mbox{}}
\fi
\ifx\subparagraph\undefined\else
\let\oldsubparagraph\subparagraph
\renewcommand{\subparagraph}[1]{\oldsubparagraph{#1}\mbox{}}
\fi

% set default figure placement to htbp
\makeatletter
\def\fps@figure{htbp}
\makeatother



\title{数据结构与算法I 作业22}
\author{2019201409 于倬浩}
\begin{document}

\maketitle
\hypertarget{header-n4}{%
\subsection{21.2-5}\label{header-n4}}

将每个元素中用来存储指向集合对象的指针指向链表的末尾即可。对于链表末尾的元素,将这个指针指向集合对象。这样,原始状态下从每个元素跳到链表末端需要先跳到集合对象,再跳到链表末端,更改后每个元素可以直接一次跳到链表末端,然后第二次跳到集合对象,不再需要集合对象中维护指向链表末端的指针。

合并时,只需将较小集合放在前面,更改每个元素指向末尾的指针到较大集合的末尾元素,最后将较小集合的末尾元素的链表指针改为较大集合的表头即可,运行时间依旧是关于较小集合元素个数的线性函数。

\hypertarget{header-n11}{%
\subsection{21.3-3}\label{header-n11}}

为了便于分析,假设\(m\)大于\(2n\)。调用\(n\)次\texttt{Make-Set}新建\(n\)个单元素集合。接下来,使用\(n - 1\)次\texttt{Union},每次将元素数目最小的两个集合合并起来,最终得到一棵深度为\(\Theta(lgn)\)的树。对于剩下的操作,任意找一个叶子节点不断进行\texttt{Find-Set},这样每次\texttt{Find-Set}的运行时间均为\(\Theta(lgn)\),这也是一次\texttt{Find-Set}操作的运行时间上界。这样渐进意义上,如果\(m\)远大于\(n\),可以确定运行时间的上界为\(\Omega(mlgn)\)。

对于\(m \leq n\)的情况,或是\(m\)与\(n\)数量级接近的情况,只需要减少\texttt{Union}的次数,例如运行\(\frac{m}{2}\)次合并,得到若干棵深度为\(O(lgn)\)的树,随便找一个叶子节点不断\texttt{Find-Set},依旧可以确定运行时间上界为\(\Omega(mlgn)\)。

\end{document}
