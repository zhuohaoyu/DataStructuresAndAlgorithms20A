\PassOptionsToPackage{unicode=true}{hyperref} % options for packages loaded elsewhere
\PassOptionsToPackage{hyphens}{url}
%
\documentclass[]{article}
\usepackage[a4paper,top=3cm,bottom=3cm,left=3cm,right=3cm,marginparwidth=1.75cm]{geometry}
\usepackage[UTF8]{ctex}
\usepackage{lmodern}
\usepackage{amssymb,amsmath}
\usepackage{float}
\usepackage{ifxetex,ifluatex}
\usepackage{fixltx2e} % provides \textsubscript
\usepackage[ruled,vlined]{algorithm2e}

\ifnum 0\ifxetex 1\fi\ifluatex 1\fi=0 % if pdftex
  \usepackage[T1]{fontenc}
  \usepackage[utf8]{inputenc}
  \usepackage{textcomp} % provides euro and other symbols
\else % if luatex or xelatex
  \usepackage{unicode-math}
  \defaultfontfeatures{Ligatures=TeX,Scale=MatchLowercase}
\fi
% use upquote if available, for straight quotes in verbatim environments
\IfFileExists{upquote.sty}{\usepackage{upquote}}{}
% use microtype if available
\IfFileExists{microtype.sty}{%
\usepackage[]{microtype}
\UseMicrotypeSet[protrusion]{basicmath} % disable protrusion for tt fonts
}{}
\IfFileExists{parskip.sty}{%
\usepackage{parskip}
}{% else
\setlength{\parindent}{0pt}
\setlength{\parskip}{6pt plus 2pt minus 1pt}
}
\usepackage{hyperref}
\hypersetup{
            pdfborder={0 0 0},
            breaklinks=true}
\urlstyle{same}  % don't use monospace font for urls
\usepackage{color}
\usepackage{fancyvrb}
\newcommand{\VerbBar}{|}
\newcommand{\VERB}{\Verb[commandchars=\\\{\}]}
\DefineVerbatimEnvironment{Highlighting}{Verbatim}{commandchars=\\\{\}}
% Add ',fontsize=\small' for more characters per line
\newenvironment{Shaded}{}{}
\newcommand{\AlertTok}[1]{\textcolor[rgb]{1.00,0.00,0.00}{\textbf{#1}}}
\newcommand{\AnnotationTok}[1]{\textcolor[rgb]{0.38,0.63,0.69}{\textbf{\textit{#1}}}}
\newcommand{\AttributeTok}[1]{\textcolor[rgb]{0.49,0.56,0.16}{#1}}
\newcommand{\BaseNTok}[1]{\textcolor[rgb]{0.25,0.63,0.44}{#1}}
\newcommand{\BuiltInTok}[1]{#1}
\newcommand{\CharTok}[1]{\textcolor[rgb]{0.25,0.44,0.63}{#1}}
\newcommand{\CommentTok}[1]{\textcolor[rgb]{0.38,0.63,0.69}{\textit{#1}}}
\newcommand{\CommentVarTok}[1]{\textcolor[rgb]{0.38,0.63,0.69}{\textbf{\textit{#1}}}}
\newcommand{\ConstantTok}[1]{\textcolor[rgb]{0.53,0.00,0.00}{#1}}
\newcommand{\ControlFlowTok}[1]{\textcolor[rgb]{0.00,0.44,0.13}{\textbf{#1}}}
\newcommand{\DataTypeTok}[1]{\textcolor[rgb]{0.56,0.13,0.00}{#1}}
\newcommand{\DecValTok}[1]{\textcolor[rgb]{0.25,0.63,0.44}{#1}}
\newcommand{\DocumentationTok}[1]{\textcolor[rgb]{0.73,0.13,0.13}{\textit{#1}}}
\newcommand{\ErrorTok}[1]{\textcolor[rgb]{1.00,0.00,0.00}{\textbf{#1}}}
\newcommand{\ExtensionTok}[1]{#1}
\newcommand{\FloatTok}[1]{\textcolor[rgb]{0.25,0.63,0.44}{#1}}
\newcommand{\FunctionTok}[1]{\textcolor[rgb]{0.02,0.16,0.49}{#1}}
\newcommand{\ImportTok}[1]{#1}
\newcommand{\InformationTok}[1]{\textcolor[rgb]{0.38,0.63,0.69}{\textbf{\textit{#1}}}}
\newcommand{\KeywordTok}[1]{\textcolor[rgb]{0.00,0.44,0.13}{\textbf{#1}}}
\newcommand{\NormalTok}[1]{#1}
\newcommand{\OperatorTok}[1]{\textcolor[rgb]{0.40,0.40,0.40}{#1}}
\newcommand{\OtherTok}[1]{\textcolor[rgb]{0.00,0.44,0.13}{#1}}
\newcommand{\PreprocessorTok}[1]{\textcolor[rgb]{0.74,0.48,0.00}{#1}}
\newcommand{\RegionMarkerTok}[1]{#1}
\newcommand{\SpecialCharTok}[1]{\textcolor[rgb]{0.25,0.44,0.63}{#1}}
\newcommand{\SpecialStringTok}[1]{\textcolor[rgb]{0.73,0.40,0.53}{#1}}
\newcommand{\StringTok}[1]{\textcolor[rgb]{0.25,0.44,0.63}{#1}}
\newcommand{\VariableTok}[1]{\textcolor[rgb]{0.10,0.09,0.49}{#1}}
\newcommand{\VerbatimStringTok}[1]{\textcolor[rgb]{0.25,0.44,0.63}{#1}}
\newcommand{\WarningTok}[1]{\textcolor[rgb]{0.38,0.63,0.69}{\textbf{\textit{#1}}}}
\usepackage{graphicx,grffile}
\makeatletter
\def\maxwidth{\ifdim\Gin@nat@width>\linewidth\linewidth\else\Gin@nat@width\fi}
\def\maxheight{\ifdim\Gin@nat@height>\textheight\textheight\else\Gin@nat@height\fi}
\makeatother
% Scale images if necessary, so that they will not overflow the page
% margins by default, and it is still possible to overwrite the defaults
% using explicit options in \includegraphics[width=8cm][width, height, ...]{}
\setkeys{Gin}{width=\maxwidth,height=\maxheight,keepaspectratio}
\setlength{\emergencystretch}{3em}  % prevent overfull lines
\providecommand{\tightlist}{%
  \setlength{\itemsep}{0pt}\setlength{\parskip}{0pt}}
\setcounter{secnumdepth}{0}
% Redefines (sub)paragraphs to behave more like sections
\ifx\paragraph\undefined\else
\let\oldparagraph\paragraph
\renewcommand{\paragraph}[1]{\oldparagraph{#1}\mbox{}}
\fi
\ifx\subparagraph\undefined\else
\let\oldsubparagraph\subparagraph
\renewcommand{\subparagraph}[1]{\oldsubparagraph{#1}\mbox{}}
\fi

% set default figure placement to htbp
\makeatletter
\def\fps@figure{htbp}
\makeatother

\title{数据结构与算法I 作业21}
\author{2019201409 于倬浩}
\begin{document}

\maketitle

\hypertarget{header-n65}{%
\subsection{18.2-1}\label{header-n65}}

\begin{figure}[H]
\centering
\includegraphics[width=8cm]{C:/Users/zhuoh/Desktop/Docs/ds-21/数据结构与算法I 作业21.assets/image-20201223105222417.png}
\caption{插入Q后}
\end{figure}


\begin{figure}[H]
\centering
\includegraphics[width=8cm]{C:/Users/zhuoh/Desktop/Docs/ds-21/数据结构与算法I 作业21.assets/image-20201223105502360.png}
\caption{插入K后}
\end{figure}

\begin{figure}[H]
\centering
\includegraphics[width=8cm]{C:/Users/zhuoh/Desktop/Docs/ds-21/数据结构与算法I 作业21.assets/image-20201223105544987.png}
\caption{插入C后}
\end{figure}


\begin{figure}[H]
\centering
\includegraphics[width=8cm]{C:/Users/zhuoh/Desktop/Docs/ds-21/数据结构与算法I 作业21.assets/image-20201223105836971.png}
\caption{插入L后}
\end{figure}


\begin{figure}[H]
\centering
\includegraphics[width=8cm]{C:/Users/zhuoh/Desktop/Docs/ds-21/数据结构与算法I 作业21.assets/image-20201223110130832.png}
\caption{插入V后}
\end{figure}

\begin{figure}[H]
\centering
\includegraphics[width=8cm]{C:/Users/zhuoh/Desktop/Docs/ds-21/数据结构与算法I 作业21.assets/image-20201223110307641.png}
\caption{插入W后}
\end{figure}


\begin{figure}[H]
\centering
\includegraphics[width=8cm]{C:/Users/zhuoh/Desktop/Docs/ds-21/数据结构与算法I 作业21.assets/image-20201223110946864.png}
\caption{插入M前(分裂)}
\end{figure}


\begin{figure}[H]
\centering
\includegraphics[width=8cm]{C:/Users/zhuoh/Desktop/Docs/ds-21/数据结构与算法I 作业21.assets/image-20201223111138326.png}
\caption{插入M后}
\end{figure}

\begin{figure}[H]
\centering
\includegraphics[width=8cm]{C:/Users/zhuoh/Desktop/Docs/ds-21/数据结构与算法I 作业21.assets/image-20201223111514022.png}
\caption{插入P后}
\end{figure}


\begin{figure}[H]
\centering
\includegraphics[width=8cm]{C:/Users/zhuoh/Desktop/Docs/ds-21/数据结构与算法I 作业21.assets/image-20201223112013966.png}
\caption{插入A前(分裂)}
\end{figure}


\begin{figure}[H]
\centering
\includegraphics[width=8cm]{C:/Users/zhuoh/Desktop/Docs/ds-21/数据结构与算法I 作业21.assets/image-20201223112319694.png}
\caption{插入Y前(分裂)}
\end{figure}


\begin{figure}[H]
\centering
\includegraphics[width=8cm]{C:/Users/zhuoh/Desktop/Docs/ds-21/数据结构与算法I 作业21.assets/image-20201223112618688.png}
\caption{最终结果}
\end{figure}

\hypertarget{header-n105}{%
\subsection{18.3-2}\label{header-n105}}

\begin{Shaded}
  \begin{Highlighting}[]
  \DataTypeTok{void}\NormalTok{ B_Tree_Delete(Node *x, }\DataTypeTok{int}\NormalTok{ k) \{}
  \NormalTok{    B_Tree_Read(x);}
      \ControlFlowTok{if}\NormalTok{(x->leaf == }\KeywordTok{true}\NormalTok{) }
          \KeywordTok{delete}\NormalTok{ x->key[k];}
      \ControlFlowTok{else}\NormalTok{ \{}
          \ControlFlowTok{if}\NormalTok{(k in x->key[]) \{}
  \NormalTok{            node *&y = find_predecessor(x, k), *&z = find_successor(x, k);}
  \NormalTok{            node *kk = find_child(x, k);}
              \ControlFlowTok{if}\NormalTok{(y->size >= t) \{}
  \NormalTok{                node *yy = find_predecessor(y, k);}
  \NormalTok{                B_Tree_Delete(yy, k);}
  \NormalTok{                y = yy;}
  \NormalTok{            \}}
              \ControlFlowTok{else}\NormalTok{ \{}
                  \ControlFlowTok{if}\NormalTok{(z->size >= t) \{}
  \NormalTok{                    node *zz = find_successor(z, k);}
  \NormalTok{                    B_Tree_Delete(zz, k);}
  \NormalTok{                    z = zz;}
  \NormalTok{                \}}
                  \ControlFlowTok{else}\NormalTok{ \{}
                      \ControlFlowTok{if}\NormalTok{(z->size + y->size == }\DecValTok{2}\NormalTok{*t - }\DecValTok{1}\NormalTok{) \{}
  \NormalTok{                        merge(y, kk, z);}
  \NormalTok{                        x->key[].erase(kk), x->key.erase(z);}
  \NormalTok{                        B_Tree_Delete(y, k);}
  \NormalTok{                    \}}
  \NormalTok{                \}}
  \NormalTok{            \}}
  \NormalTok{        \}}
          \ControlFlowTok{else}\NormalTok{ \{}
  \NormalTok{            node *xk = Find_K_recursively(x, k), *xkf = xk;}
  \NormalTok{            node *last = Find_father_of_xk(x, k);}
              \ControlFlowTok{if}\NormalTok{(xk->size == t - }\DecValTok{1}\NormalTok{ && (xk->left->size >= t || xk->right->size >= t)) \{}
  \NormalTok{                xk->key.insert(last);}
                  \ControlFlowTok{if}\NormalTok{(xk->left->size >= t) x->key.insert(xk->left);}
                  \ControlFlowTok{else}\NormalTok{ x->key.insert(xk->right);}
  \NormalTok{            \}}
              \ControlFlowTok{if}\NormalTok{(xk->size == t - }\DecValTok{1}\NormalTok{ && xk->left->size == t - }\DecValTok{1}\NormalTok{ && xk->right->size == t - }\DecValTok{1}\NormalTok{) \{}
  \NormalTok{                merge(xk->left, last, xk);}
  \NormalTok{            \}}
  \NormalTok{            B_Tree_Delete(last, k);}
  \NormalTok{        \}}
          
  \NormalTok{    \}}
  \NormalTok{    B_Tree_Write(x);}
  \NormalTok{\}}
  \end{Highlighting}
  \end{Shaded}


% \begin{algorithm}[H]
%   \SetAlgoLined
%     B-Tree-Read(x)\; 
%     \eIf{x->leaf == true} {
%       delete x->key[k];\
%     }{
%       \eIf{k in x->key[]} {
%         node *y = find_predecessor(x, k) (找到k在儿子指针中的前驱)\;
%       }{
  
%       }
%     }
%     B-Tree-Write(x)\;
%     \caption{B-Tree-Delete(node x, data k)}
%   \end{algorithm}
  

\end{document}
