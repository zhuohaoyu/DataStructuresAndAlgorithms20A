\PassOptionsToPackage{unicode=true}{hyperref} % options for packages loaded elsewhere
\PassOptionsToPackage{hyphens}{url}
%
\documentclass[]{article}
\usepackage[a4paper,top=3cm,bottom=3cm,left=3cm,right=3cm,marginparwidth=1.75cm]{geometry}
\usepackage[UTF8]{ctex}
\usepackage{lmodern}
\usepackage{amssymb,amsmath}
\usepackage{ifxetex,ifluatex}
\usepackage{fixltx2e} % provides \textsubscript
\ifnum 0\ifxetex 1\fi\ifluatex 1\fi=0 % if pdftex
  \usepackage[T1]{fontenc}
  \usepackage[utf8]{inputenc}
  \usepackage{textcomp} % provides euro and other symbols
\else % if luatex or xelatex
  \usepackage{unicode-math}
  \defaultfontfeatures{Ligatures=TeX,Scale=MatchLowercase}
\fi
% use upquote if available, for straight quotes in verbatim environments
\IfFileExists{upquote.sty}{\usepackage{upquote}}{}
% use microtype if available
\IfFileExists{microtype.sty}{%
\usepackage[]{microtype}
\UseMicrotypeSet[protrusion]{basicmath} % disable protrusion for tt fonts
}{}
\IfFileExists{parskip.sty}{%
\usepackage{parskip}
}{% else
\setlength{\parindent}{0pt}
\setlength{\parskip}{6pt plus 2pt minus 1pt}
}
\usepackage{hyperref}
\hypersetup{
            pdfborder={0 0 0},
            breaklinks=true}
\urlstyle{same}  % don't use monospace font for urls
\usepackage{color}
\usepackage{fancyvrb}
\newcommand{\VerbBar}{|}
\newcommand{\VERB}{\Verb[commandchars=\\\{\}]}
\DefineVerbatimEnvironment{Highlighting}{Verbatim}{commandchars=\\\{\}}
% Add ',fontsize=\small' for more characters per line
\newenvironment{Shaded}{}{}
\newcommand{\AlertTok}[1]{\textcolor[rgb]{1.00,0.00,0.00}{\textbf{#1}}}
\newcommand{\AnnotationTok}[1]{\textcolor[rgb]{0.38,0.63,0.69}{\textbf{\textit{#1}}}}
\newcommand{\AttributeTok}[1]{\textcolor[rgb]{0.49,0.56,0.16}{#1}}
\newcommand{\BaseNTok}[1]{\textcolor[rgb]{0.25,0.63,0.44}{#1}}
\newcommand{\BuiltInTok}[1]{#1}
\newcommand{\CharTok}[1]{\textcolor[rgb]{0.25,0.44,0.63}{#1}}
\newcommand{\CommentTok}[1]{\textcolor[rgb]{0.38,0.63,0.69}{\textit{#1}}}
\newcommand{\CommentVarTok}[1]{\textcolor[rgb]{0.38,0.63,0.69}{\textbf{\textit{#1}}}}
\newcommand{\ConstantTok}[1]{\textcolor[rgb]{0.53,0.00,0.00}{#1}}
\newcommand{\ControlFlowTok}[1]{\textcolor[rgb]{0.00,0.44,0.13}{\textbf{#1}}}
\newcommand{\DataTypeTok}[1]{\textcolor[rgb]{0.56,0.13,0.00}{#1}}
\newcommand{\DecValTok}[1]{\textcolor[rgb]{0.25,0.63,0.44}{#1}}
\newcommand{\DocumentationTok}[1]{\textcolor[rgb]{0.73,0.13,0.13}{\textit{#1}}}
\newcommand{\ErrorTok}[1]{\textcolor[rgb]{1.00,0.00,0.00}{\textbf{#1}}}
\newcommand{\ExtensionTok}[1]{#1}
\newcommand{\FloatTok}[1]{\textcolor[rgb]{0.25,0.63,0.44}{#1}}
\newcommand{\FunctionTok}[1]{\textcolor[rgb]{0.02,0.16,0.49}{#1}}
\newcommand{\ImportTok}[1]{#1}
\newcommand{\InformationTok}[1]{\textcolor[rgb]{0.38,0.63,0.69}{\textbf{\textit{#1}}}}
\newcommand{\KeywordTok}[1]{\textcolor[rgb]{0.00,0.44,0.13}{\textbf{#1}}}
\newcommand{\NormalTok}[1]{#1}
\newcommand{\OperatorTok}[1]{\textcolor[rgb]{0.40,0.40,0.40}{#1}}
\newcommand{\OtherTok}[1]{\textcolor[rgb]{0.00,0.44,0.13}{#1}}
\newcommand{\PreprocessorTok}[1]{\textcolor[rgb]{0.74,0.48,0.00}{#1}}
\newcommand{\RegionMarkerTok}[1]{#1}
\newcommand{\SpecialCharTok}[1]{\textcolor[rgb]{0.25,0.44,0.63}{#1}}
\newcommand{\SpecialStringTok}[1]{\textcolor[rgb]{0.73,0.40,0.53}{#1}}
\newcommand{\StringTok}[1]{\textcolor[rgb]{0.25,0.44,0.63}{#1}}
\newcommand{\VariableTok}[1]{\textcolor[rgb]{0.10,0.09,0.49}{#1}}
\newcommand{\VerbatimStringTok}[1]{\textcolor[rgb]{0.25,0.44,0.63}{#1}}
\newcommand{\WarningTok}[1]{\textcolor[rgb]{0.38,0.63,0.69}{\textbf{\textit{#1}}}}
\setlength{\emergencystretch}{3em}  % prevent overfull lines
\providecommand{\tightlist}{%
  \setlength{\itemsep}{0pt}\setlength{\parskip}{0pt}}
\setcounter{secnumdepth}{0}
% Redefines (sub)paragraphs to behave more like sections
\ifx\paragraph\undefined\else
\let\oldparagraph\paragraph
\renewcommand{\paragraph}[1]{\oldparagraph{#1}\mbox{}}
\fi
\ifx\subparagraph\undefined\else
\let\oldsubparagraph\subparagraph
\renewcommand{\subparagraph}[1]{\oldsubparagraph{#1}\mbox{}}
\fi

% set default figure placement to htbp
\makeatletter
\def\fps@figure{htbp}
\makeatother



\title{数据结构与算法I 作业20}
\author{2019201409 于倬浩}

\begin{document}
\maketitle


\hypertarget{header-n32}{%
\subsection{设计外存选择算法}\label{header-n32}}

考虑改进最坏线性时间复杂度的\texttt{Select}算法。首先,假设共有\texttt{N}个要处理的元素,内存可以容纳\texttt{M}个元素,每块存储块可容纳\texttt{B}个元素。

和原始算法类似,假设我们当前处理\texttt{Select(L,\ R,\ k)},先把这\(\Theta(N)\)个元素分成\(\frac{N}{M}\)块,每次读入\(M\)个元素至内存中,使用内存中的中位数算法以\(\Theta(M)\)的时间复杂度找到中位数并存下这些中位数,最后使用内存中位数算法算出这\(\frac{N}{M}\)个中位数的中位数\(P\)。接下来进行\texttt{partition}操作,每次读入\(B\)个元素至内存中,并维护三个缓冲块。块\texttt{A}维护了小于等于\(P\)的元素,块\(B\)维护大于\(P\)的元素,块\(C\)维护从外存中读取的当前要处理的\(B\)个元素。当前两个块有一个填满时,写入外存。写入策略和内存的\texttt{partition}算法一致,因此每次块满后,最好情况只需要一次写入,最坏情况还需要加一次整块交换。最终,当所有块已经处理完毕,只需将缓冲区中剩余的元素写入外存即可。因此,每次递归处理\texttt{Select(L,R,k)},需要\(\Theta(\frac{N}{B})\)次I/O操作。

根据朴素算法的时间复杂度分析,我们只需保证每次分的块\(\frac{N}{M}\)大于5即可保证总时间复杂度为线性,而由于元素的访问次数和时间复杂度又成线性关系,因此可以保证总的元素读写次数为\(\Theta(n)\),且由于保证了连续I/O,我们总共对外存的I/O次数也是\(\Theta(\frac{n}{B})\)。

然而,如果\(\frac{N}{M}\)过大,即要处理的元素数目远大于内存可以容纳的数量,就不再能使用内存的中位数算法算出\(\frac{N}{M}\)个数字的的中位数。因此,还需要按照朴素算法,五个数分一组,然后之后进行和上述描述的算法相同的操作。这样做的优点是不再需要关注内存大小,但是缺点是对外存的I/O次数增加了一倍(每次算出中位数后,都需要写入外存),但仍能达到要求的\(\Theta(\frac{n}{B})\)次I/O。

\newpage

\hypertarget{header-n38}{%
\subsection{Cache-Oblivious算法}\label{header-n38}}

\begin{itemize}
\item
  a. 证明Quicksort不是一个Cache-Oblivious算法

  \texttt{QuickSort}算法的主要时间消耗、数据读写主要集中于\texttt{partition}操作,因此我们只需考虑\texttt{partition}操作带来的影响。\texttt{partition}操作实际上是对要分区的区间,将大于主元的元素放在尾部,将小于主元的元素放在中部,不断迭代。虽然最终\texttt{partition}操作的区间将被不断细分,直到cache中可以存下整个当前操作的区间,但是\texttt{partition}操作本身并不是递归算法,即使当前处理的区间大于cache,依然会使用同样的方式处理。

  举例而言,假设当前\texttt{partition}的区间长度为\(\Theta(N)\)级别,不能放入cache中,该算法就会对区间内的数据不断进行无缓冲的读写,因此最坏情况下,例如给出一个极大元素、极小元素交替的数组,那么每次都需要往数组的中部/尾部交替写入,因此需要\(\Theta(N)\)次I/O,高于\(\Theta(N/B)\)。

  因此,最坏情况下,I/O的操作数至少可以达到\(O(N + \frac{N}{B}log_{\frac{M}{B}}\frac{N}{B})\),高于\(\frac{N}{B}log_{\frac{M}{B}}\frac{N}{B}\)。
\item
  b.
  考虑如下算法,证明其在不知道B和M的情况下,I/O复杂度可达到\(O(\frac{N}{B}log_{\frac{M}{B}}\frac{N}{B})\)个I/O

  该算法整体上是一个递归的过程,直到要处理的区间可以完全被放到cache中,接下来的读写均在cache中进行,因此只需考虑大于cache大小的区间,在计算的过程中是否可以保证连续I/O。

  观察算法流程,发现实际上只有前两步\texttt{partition}和\texttt{distribution}会对数据产生实际的操作。如果可以保证\texttt{partition}和\texttt{distribution}的读写连续性,即使不知道\(B\)依然可以做到单次调用产生的I/O次数为\(O(N/B)\)。

  因此,如果我们保证了\texttt{distribution}的读写连续性,每层递归的区间长度都是上一层的\(\frac{1}{\sqrt{N}}倍\),而当递归深度到\(O(log_{\frac{M}{B}}\frac{N}{B})\)时,显然此时的区间已经足够小,可以直接放在cache中了,因此更深层次的递归不会带来外存I/O操作数目的增加。因此,每一层都需要对整个区间进行读写,而最多有\(O(log_{\frac{M}{B}}\frac{N}{B})\)层,因此该算法的I/O操作数为\(O(\frac{N}{B}log_{\frac{M}{B}}\frac{N}{B})\)。
\item
  c. 给出符合上述要求的distribution过程伪代码

\begin{Shaded}
\begin{Highlighting}[]
\DataTypeTok{void}\NormalTok{ distribute(}\DataTypeTok{int}\NormalTok{ L, }\DataTypeTok{int}\NormalTok{ R, }\DataTypeTok{int}\NormalTok{ P) \{}
    \CommentTok{// 将编号为[L, R]范围内的子数组,插进从P开始的若干个桶}
    \ControlFlowTok{if}\NormalTok{(L == R) \{}
\NormalTok{        Load Array[L] into RAM;}
\NormalTok{        Initialize RAM buffer Buf[];}
        \ControlFlowTok{for}\NormalTok{(}\KeywordTok{auto}\NormalTok{ i: Array[L]) \{ }\CommentTok{//连续访问第i个子数组,}
            \ControlFlowTok{if}\NormalTok{(i < B[P].pivot) Add i to Buf[];}
\NormalTok{        \}}
        \CommentTok{//使用一次外存读写将整个数组合并起来}
\NormalTok{        Concat(B[P], Buf);}
        
		\CommentTok{// 如果大小不满足要求,split操作将一个桶分裂成两个。}
        \ControlFlowTok{if}\NormalTok{(B[P].size > }\DecValTok{2}\NormalTok{ * sqrt(N)) split(B[P]); }
        \CommentTok{// split()使用线性中位数算法和partition算法分割区间,}
        \CommentTok{// 在分裂后, 重新计算两个桶pivot,}
        \CommentTok{// 较大的桶pivot为B[P].pivot, 较小的为找到的中位数。}
\NormalTok{    \}}
    \ControlFlowTok{else}\NormalTok{ \{}
        \DataTypeTok{int}\NormalTok{ M = (L + R) / }\DecValTok{2}\NormalTok{; }
        \CommentTok{//递归处理,将前一半/后一半的子数组中较大/较小元素分成两部分}
\NormalTok{        distribute(L, M, P);}
\NormalTok{        distribute(M + }\DecValTok{1}\NormalTok{, R, P);}
\NormalTok{        distribute(L, M, P + (R - L + }\DecValTok{1}\NormalTok{) / }\DecValTok{2}\NormalTok{);}
\NormalTok{        distribute(M + }\DecValTok{1}\NormalTok{, R, P + (R - L + }\DecValTok{1}\NormalTok{) / }\DecValTok{2}\NormalTok{);}
\NormalTok{    \}}
    
\NormalTok{\}}
\end{Highlighting}
\end{Shaded}

  这道题有阅读参考文献,自己真的是想不到这么妙的东西,但是发现参考文献上的伪代码并没有给出关键细节(如何保证连续读写),于是补上了文献中缺失的部分。

  参考资料:\href{http://supertech.csail.mit.edu/papers/Prokop99.pdf}{Cache-Oblivious
  Algorithms. Harald Prokop. MIT}
\end{itemize}

\end{document}
