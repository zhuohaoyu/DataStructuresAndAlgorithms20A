\PassOptionsToPackage{unicode=true}{hyperref} % options for packages loaded elsewhere
\PassOptionsToPackage{hyphens}{url}
%
\documentclass{article}
\usepackage[a4paper, total={6in, 8in}]{geometry}
\usepackage{ctex}
\usepackage{lmodern}
\usepackage{amssymb,amsmath}
\usepackage{ifxetex,ifluatex}
\usepackage{fixltx2e} % provides \textsubscript
\ifnum 0\ifxetex 1\fi\ifluatex 1\fi=0 % if pdftex
  \usepackage[T1]{fontenc}
  \usepackage[utf8]{inputenc}
  \usepackage{textcomp} % provides euro and other symbols
\else % if luatex or xelatex
  \usepackage{unicode-math}
  \defaultfontfeatures{Ligatures=TeX,Scale=MatchLowercase}
\fi
% use upquote if available, for straight quotes in verbatim environments
\IfFileExists{upquote.sty}{\usepackage{upquote}}{}
% use microtype if available
\IfFileExists{microtype.sty}{%
\usepackage[]{microtype}
\UseMicrotypeSet[protrusion]{basicmath} % disable protrusion for tt fonts
}{}
\IfFileExists{parskip.sty}{%
\usepackage{parskip}
}{% else
\setlength{\parindent}{0pt}
\setlength{\parskip}{6pt plus 2pt minus 1pt}
}
\usepackage{hyperref}
\hypersetup{
            pdfborder={0 0 0},
            breaklinks=true}
\urlstyle{same}  % don't use monospace font for urls
\setlength{\emergencystretch}{3em}  % prevent overfull lines
\providecommand{\tightlist}{%
  \setlength{\itemsep}{0pt}\setlength{\parskip}{0pt}}
\setcounter{secnumdepth}{0}
% Redefines (sub)paragraphs to behave more like sections
\ifx\paragraph\undefined\else
\let\oldparagraph\paragraph
\renewcommand{\paragraph}[1]{\oldparagraph{#1}\mbox{}}
\fi
\ifx\subparagraph\undefined\else
\let\oldsubparagraph\subparagraph
\renewcommand{\subparagraph}[1]{\oldsubparagraph{#1}\mbox{}}
\fi

% set default figure placement to htbp
\makeatletter
\def\fps@figure{htbp}
\makeatother

\title{数据结构与算法I 作业17}
\author{2019201409 于倬浩}

\begin{document}
\maketitle

\hypertarget{header-n33}{%
\subsection{17.1-3 聚合分析}\label{header-n33}}

设\(n\)次操作的总运行时间为\(T(n)\),则有:

\[\begin{aligned}
	T(n) & = \sum_{i=1}^{n} [i = 2^k] i + [i \neq 2^k] 1 \\
	& = \sum_{i=1}^{\lfloor lg(n) \rfloor} 2^i + \sum_{i=1}^{n}1 - \lfloor lg(n) \rfloor \\
	& = \sum_{i=1}^{\lfloor lg(n) \rfloor} 2^i + n - lg(n) \\
	& \leq 2n + n - lg(n) \\
	& = 3n - lg(n) \\
	& = O(n)
\end{aligned}\]

因此,单次操作的运行时间则为\(T(n)/n = O(n) / n = O(1)\)。

\hypertarget{header-n7}{%
\subsection{17.2-2 核算分析}\label{header-n7}}

设第\(i\)个操作的实际代价为\(c_i\),摊还代价为\(\hat{c_i}\),有:

\[\begin{aligned}
	c_i & = \begin{cases}
				1 & (i \neq 2^k) \\
				i & (i = 2^k)
			\end{cases}
\end{aligned}\]

令\(\hat{c_i} = 3\),即令每个操作的摊还代价为3。每个操作若实际代价为1,则存起来2的credit;如果实际代价大于1,则开始消耗之前存起来的credit。

根据\texttt{17.1-3},对任意n都有:

\[\begin{aligned}
	\sum_{i=1}^{n}c_i & \leq 3n-lg(n) \\
	& \leq 3n \\
	& \leq \sum_{i=1}^{n} \hat{c_i}
\end{aligned}\]

因此n个操作的总代价为\(O(n)\)。

\hypertarget{header-n11}{%
\subsection{17.3-2 势能分析}\label{header-n11}}

设\(D_i\)表示第\(i\)次操作后的状态,\(\Phi(D_i)\)为此状态的势函数。

首先,使用势能分析,要满足:

\[\begin{aligned}
	\hat{c_i} &= c_i + \Phi(D_i) - \Phi(D_{i-1}) \\
	\sum_{i=1}^{n} \hat{c_i} &\ge \sum_{i=1}^{n} c_i
\end{aligned}\]

因此,尝试构造\(\Phi\)如下:

\[\begin{aligned}
	\Phi(D_i) = \begin{cases} 
		0 \; & i=0 \\
		i - (2^{\lfloor lg(i) \rfloor  + 1}) & i \neq 0 \\
	\end{cases}
\end{aligned}\]

这种构造是利用势能分析的实际意义定义的。当\(i=2^k\)时,\(\Phi(D_i)-\Phi(D_{i-1}) = i - 2^{k+1} - ((i-1) - 2^{k}) = 1-2^k = 1 - i\),而当\(i \neq 2^k\)时,\(\Phi(D_i) - \Phi(D_{i-1}) = 1\)。这两点可以符合势能差的实际含义,但是并不能满足\(\Phi(D_i) \ge 0\)恒成立,因此修改构造如下:

\[\begin{aligned}
	\Phi(D_i) = \begin{cases} 
		0 \; & i=0 \\
		2i - (2^{\lfloor lg(i) \rfloor  + 1}) - 1 & i \neq 0 \\
	\end{cases}
\end{aligned}\]

这样,可以保证势函数恒为正,势能差满足:

\[\begin{aligned}
	\Phi(D_i) - \Phi(D_{i-1}) = \begin{cases}
		2 - i \; & i = 2^k \\
		2 \; & i \ne 2^k
	\end{cases}
\end{aligned}\]

因此有:

\[\begin{aligned}
	\sum_{i=1}^{n} \hat{c_i} &= \sum_{i=1}^{n}c_i + \Phi(D_i)  - \Phi(D_{i-1})\\
    &= \sum_{i=1} ^{n} [i=2^k](2-i + i) + [i \ne 2^k](2+1) \\
	& \leq 3n
\end{aligned}\]

因此,\(n\)次操作的运行时间为\(O(n)\),单次操作摊还代价为\(O(1)\)。

\end{document}
