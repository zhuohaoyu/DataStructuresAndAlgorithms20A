\PassOptionsToPackage{unicode=true}{hyperref} % options for packages loaded elsewhere
\PassOptionsToPackage{hyphens}{url}
%
\documentclass[]{article}
\usepackage[a4paper,top=3cm,bottom=3cm,left=3cm,right=3cm,marginparwidth=1.75cm]{geometry}
\usepackage[UTF8]{ctex}
\usepackage{lmodern}
\usepackage{amssymb,amsmath}
\usepackage{ifxetex,ifluatex}
\usepackage{fixltx2e} % provides \textsubscript
\ifnum 0\ifxetex 1\fi\ifluatex 1\fi=0 % if pdftex
  \usepackage[T1]{fontenc}
  \usepackage[utf8]{inputenc}
  \usepackage{textcomp} % provides euro and other symbols
\else % if luatex or xelatex
  \usepackage{unicode-math}
  \defaultfontfeatures{Ligatures=TeX,Scale=MatchLowercase}
\fi
% use upquote if available, for straight quotes in verbatim environments
\IfFileExists{upquote.sty}{\usepackage{upquote}}{}
% use microtype if available
\IfFileExists{microtype.sty}{%
\usepackage[]{microtype}
\UseMicrotypeSet[protrusion]{basicmath} % disable protrusion for tt fonts
}{}
\IfFileExists{parskip.sty}{%
\usepackage{parskip}
}{% else
\setlength{\parindent}{0pt}
\setlength{\parskip}{6pt plus 2pt minus 1pt}
}
\usepackage{hyperref}
\hypersetup{
            pdfborder={0 0 0},
            breaklinks=true}
\urlstyle{same}  % don't use monospace font for urls
\setlength{\emergencystretch}{3em}  % prevent overfull lines
\providecommand{\tightlist}{%
  \setlength{\itemsep}{0pt}\setlength{\parskip}{0pt}}
\setcounter{secnumdepth}{0}
% Redefines (sub)paragraphs to behave more like sections
\ifx\paragraph\undefined\else
\let\oldparagraph\paragraph
\renewcommand{\paragraph}[1]{\oldparagraph{#1}\mbox{}}
\fi
\ifx\subparagraph\undefined\else
\let\oldsubparagraph\subparagraph
\renewcommand{\subparagraph}[1]{\oldsubparagraph{#1}\mbox{}}
\fi

% set default figure placement to htbp
\makeatletter
\def\fps@figure{htbp}
\makeatother


\title{数据结构与算法I 作业19}
\author{2019201409 于倬浩}

\begin{document}

\maketitle

\hypertarget{header-n4}{%
\subsection{19.4-1}\label{header-n4}}

由于每次\texttt{Decrease\_Key}都会从原堆中砍掉以该节点为根的一棵子树,因此考虑使用此操作构造定长的链。

首先,假设我们能构造出一条长度为\(k\)且每个节点都没被打标记的链,我们如果可以再通过同样的方法,构造出来一条性质完全相同,但是链顶部节点的键值小于当前链顶节点的键值,那么我们可以通过随便插入一个值更小的节点,然后调用\texttt{Extract\_Min}来删除这个值更小的节点,顺便引发一次\texttt{Consolidate}来调整树的形态,得到的结果就是一颗由两条链构成的树,最长的一条链的长度为\(k+1\),另一条链长度为\(k\),那么我们通过\texttt{Decrease\_Key}砍掉另一条链上的\(k-1\)个节点,并通过\(k-1\)次\texttt{Decrease\_Key}和\texttt{Extract\_Min}删掉这\(k-1\)个节点,即可得到一个长度为\(k+1\)且根节点被标记的链。

接下来,我们继续采用同样的方法,构造出另一条长度为\(k+1\)的链,但是根节点被标记。再次随便插入一个更小的值,调用一次\texttt{Extract\_Min},触发一次\texttt{Consolidate},拼成一颗两条链构成的树,最长链长度为\(k+2\),另一条链长度为\(k+1\),但是树根和次长链方向的节点(共两个节点)有标记。因此,我们对次长链方向有标记的节点使用\texttt{Decrease\_Key},由于\texttt{Cascading\_Cut}操作,原树根和当前节点都被切掉。

最后,只需要使用\(k+1\)次的删除操作,切掉多余的节点,即可得到一个长度为\(k+1\)的链,且每个节点都没有标记。

对于边界情况,长度为\(1\)的链,只需要插入\(3\)个节点,使用一次删除操作除掉最小节点即可得到。

梳理一下思路,我们实际上是通过数学归纳法,如果存在构造长度为\(k\)的无标记节点构成的链的方法,那么就可以构造出长度为\(k+1\)的无标记节点构成的链,且对于边界情况\(k=1\)仍然可行,因此我们可以构造出任意由\(k\)个节点构成的链。注意到这种方法在构造的过程中,已经证明了正确性。

\hypertarget{header-n41}{%
\subsection{19-1}\label{header-n41}}

\begin{itemize}
\item
  a.

  由于每个结点的度数都是\(O(lgn)\)级别的,而不是\(O(1)\)级别的。将\(x\)的儿子直接放到根链表后,还需要逐个更新儿子们的父亲指针,因此需要节点度数次运算,因此第七行操作的分析不正确,实际运行时间为\(O(lgn)\)(或认为是\(O(x.degree)\))。
\item
  b.

  由于每次\texttt{Cascading-Cut}的运行时间为\(O(1)\),共需运行\(c\)次,且第七行代码的运行时间为\(O(x.degree)\),因此实际运行时间为\(O(c+x.degree)\)
\item
  c.

  考虑势函数的增量。

  该次操作执行了\(c\)次\texttt{Cascading\_Cut},最多切出\(c-1\)棵树放进根链表,最多将1个节点由未标记变成标记,\texttt{Pisano\_Delete}的第7行会增加\(x.degree\)棵树,因此操作过后,树的总数最多为\(t(H)+c-1+x.degree\),被标记的节点总数最多增加\(1\),但是一定会减少\(c\)个。因此,形式化地表示如下:

  \[\begin{aligned}
  	t(H') &\leq t(H) + c - 1 + x.degree \\
  	m(H') &\leq m(H) - c + 1 \\
  	\therefore \Phi(H') & \leq t(H)+2m(H)+x.degree+1
  \end{aligned}\]
\item
  d.

  和原算法一样,最终的摊还代价依旧是\(\Theta(x.degree)\),而由于树本身的性质不变,\(\Theta(x.degree)=\Theta(lgn)\)这一性质不变,因此算法摊还代价渐进意义上和原来保持一致。
\end{itemize}

\end{document}
